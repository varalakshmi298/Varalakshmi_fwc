\documentclass{article}
\usepackage{amsmath}
\usepackage{tikz}
\usepackage{array}
\usepackage{booktabs}
\usepackage{graphicx} % Required for inserting images
\usepackage[a4paper,
            top = 2in, 
            bottom = 1in, 
            left = 1in, 
            right=1in, 
            margin width = 1pt] {geometry}
\usepackage{fancyhdr}
\pagestyle{empty} % Start with empty page style
\thispagestyle{fancy} % Apply fancy style only to the first page
\renewcommand{\headrulewidth}{0pt} % Remove header rule
\fancyhead[L]{% Left header
        \includegraphics[width=8cm, height=1.7cm]{Question2.jpeg} 
        }
\fancyhead[R]{% Right header
    Name: N.VaraLakshmi   \\
    Batch: COMETFWC021 \\
    Date: 18 May 2025\\
}

\begin{document}

\section*{Question 1}

\textbf{Q.} The Boolean expression for the output $f$ of the multiplexer shown below is:

\begin{itemize}
  \item[(A)] $(P \oplus Q \oplus R)'$
  \item[(B)] $P \oplus Q \oplus R$
  \item[(C)] $P + Q + R$
  \item[(D)] $(P + Q + R)'$
\end{itemize}

\subsection*{Solution}

This is a 4:1 multiplexer with select lines $P$ (MSB) and $Q$ (LSB). The data inputs are connected as:

\begin{center}
\begin{tabular}{cc|c}
\toprule
$P$ & $Q$ & Selected Input \\
\midrule
0 & 0 & $R$ \\
0 & 1 & $R'$ \\
1 & 0 & $R'$ \\
1 & 1 & $R$ \\
\bottomrule
\end{tabular}
\end{center}

So, the output function becomes:
\[
f = (\overline{P} \cdot \overline{Q} \cdot R) + (\overline{P} \cdot Q \cdot R') + (P \cdot \overline{Q} \cdot R') + (P \cdot Q \cdot R)
\]

Group and simplify:
\[
f = R(\overline{P}\,\overline{Q} + P Q) + R'(\overline{P} Q + P \overline{Q})
\]

This gives:
\[
f = R \cdot \text{XNOR}(P, Q) + R' \cdot \text{XOR}(P, Q)
\Rightarrow f = P \oplus Q \oplus R
\]

\textbf{Hence, the correct answer is (B).}

\end{document}
