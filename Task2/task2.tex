\documentclass{article}
\usepackage[a4paper,
            top = 2in, 
            bottom = 0.5in, 
            left = 1in, 
            right=1in, 
            margin width = 1pt] {geometry}
\usepackage{amsfonts,amssymb,amsmath}
\usepackage{graphicx}
\usepackage{tikz}
\usepackage{subcaption}
\usepackage{array}
\usepackage{lipsum}
\usepackage{fancyhdr}
\usepackage{lmodern}
\usepackage{makecell} 
\usepackage{xcolor}
\usepackage{ragged2e}
\usepackage{enumitem}
\usepackage{newtxtext,newtxmath} 
\setlength{\parindent}{0pt}
\renewcommand{\baselinestretch}{1.35}
\usepackage{setspace}
\definecolor{ncertcyan}{HTML}{00B0F0}

\fancyfoot[C]{\small 2019-20} % Centered footer
\renewcommand{\headrule}{\color{ncertcyan}\hrule height 3pt}
\begin{document}
\pagestyle{empty} % Start with empty page style
\thispagestyle{fancy} % Apply fancy style only to the first page
\renewcommand{\headrulewidth}{0pt} % Remove header rule
\fancyhead[L]{% Left header
        \includegraphics[width=8cm, height=1.7cm]{i.png} 
        }
\fancyhead[R]{% Right header
    Name: N.VaraLakshmi \\
    Batch: COMETFWC021 \\
    Date: 18 May 2025
}
.
\vspace{1em}

are (i) intersecting, then $\frac{a_1}{a_2} \ne \frac{b_1}{b_2},$  
\\

\hspace{1.7em}(ii) coincident, then   $\frac{a_1}{a_2} = \frac{b_1}{b_2} = \frac{c_1}{c_2},$
\\

\hspace{1.7em}(iii) parallel, then $\frac{a_1}{a_2} = \frac{b_1}{b_2} \ne \frac{c_1}{c_2}.$
\vspace{1em}
\\
\par In fact, the converse is also true for any pair of lines. You can verify them by considering some more examples by yourself.

Let us now consider some more examples to illustrate it.

\noindent
\textbf{\textcolor{ncertcyan}{Example 4 :}} Check graphically whether the pair of equations
\begin{align}
x + 3y &= 6 \hspace{4cm}  \\
2x - 3y &= 12 \hspace{3.9cm}
\end{align}

\noindent
is consistent. If so, solve them graphically.

\noindent
\textbf{\textcolor{ncertcyan}{Solution :}} Let us draw the graphs of the Equations (1) and (2). For this, we find two solutions of each of the equations, which are given in Table 3.5.

\begin{center}
\textbf{\textcolor{ncertcyan}{Table 3.5}}
\end{center}
\begin{minipage}{0.45\textwidth}
    \includegraphics[width=\linewidth]{img8.jpg}
\end{minipage}
\hspace{0.05\textwidth}
\begin{minipage}{0.45\textwidth}
    \includegraphics[width=\linewidth]{img9.jpg} 
\end{minipage}

\vspace{-1em}
\noindent
\begin{minipage}{0.5\textwidth}
\hspace{2em}
{\fontsize{14}{16}\selectfont
Plot the points A(0, 2), B(6, 0), P(0, $-4$) and Q(3, $-2$) on graph paper, and join the points to form the lines AB and PQ as shown in Fig. 3.5.

\vspace{1em}
\hspace{2em}
We observe that there is a point B (6, 0) common to both the lines AB and PQ. So, the solution of the pair of linear equations is x = 6 and y = 0, i.e., the given pair of equations is consistent.}
\end{minipage}
\begin{minipage}{0.5\textwidth}
    \includegraphics[width=\linewidth]{img10.jpg}
\end{minipage}
\newpage
\pagestyle{fancy}
\fancyhf{} % Clear default headers/footers
\fancyhead[L]{\color{ncertcyan}{\Large 40\hspace{25em}\Large M\small ATHEMATICS}}
\fancyfoot[C]{\small 2019-20}
\textbf{\textcolor{ncertcyan}{Example 5:}} Graphically, find whether the following pair of equations has no solution, unique solution or infinitely many solutions:
\begin{align*}
5x - 8y + 1 &= 0 \tag{1} \\
3x - \frac{24}{5}y + \frac{3}{5} &= 0 \tag{2}
\end{align*}

\textbf{\textcolor{ncertcyan}{solution:}} Multiplying Equation (2) by $\frac{5}{3}$, we get
\[
5x - 8y + 1 = 0
\]

But, this is the same as Equation (1). Hence the lines represented by Equations (1) and (2) are coincident. Therefore, Equations (1) and (2) have infinitely many solutions.

Plot a few points on the graph and verify it yourself.

\vspace{0.5cm}
\textbf{\textcolor{ncertcyan}{Example 6:}} Champa went to a ‘Sale’ to purchase some pants and skirts. When her friends asked her how many of each she had bought, she answered, “The number of skirts is two less than twice the number of pants purchased. Also, the number of skirts is four less than four times the number of pants purchased”. Help her friends to find how many pants and skirts Champa bought.\\
\textbf{\textcolor{ncertcyan}{Solution:}} Let us denote the number of pants by $x$ and the number of skirts by $y$. Then the equations formed are:
\begin{align*}
y &= 2x - 2 \tag{1} \\
y &= 4x - 4 \tag{2}
\end{align*}

\noindent
Let us draw the graphs of Equations (1) and (2) by finding two solutions for each of the equations. They are given in Table 3.6.
\vspace{0.5cm}
\begin{center}
\textbf{\textcolor{ncertcyan}{Table 3.6}}
\end{center}

\begin{minipage}{0.45\textwidth}
    \includegraphics[width=\linewidth]{img24.jpg}
\end{minipage}
\hspace{0.05\textwidth}
\begin{minipage}{0.45\textwidth}
    \includegraphics[width=\linewidth]{img23.jpg} 
\end{minipage}
\end{document}
