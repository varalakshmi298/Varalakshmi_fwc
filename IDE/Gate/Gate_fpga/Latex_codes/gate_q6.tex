\documentclass[12pt]{article}
\usepackage{graphicx}
\usepackage[margin=1in]{geometry}
\usepackage{caption}
\usepackage{float}
\usepackage{amsmath}
\usepackage{array}
\usepackage{multicol}
\usepackage{hyperref}
\usepackage{url}

\setlength{\columnsep}{1cm}

\begin{document}

% Title Block
\begin{center}
    \textbf{\Large IMPLEMENTATION OF BOOLEAN LOGIC USING ARDUINO} \\[10pt]
    \textbf{N. VARALAKSHMI} \\
    \texttt{varalakshminissankara4@gmail.com} \\
    \textbf{COMET.FWC021} \\
    \textbf{Future Wireless Communication (FWC)} \\
    \textbf{ASSIGNMENT} \\[5pt]
    July 07, 2025
\end{center}
\vspace{1em}

\begin{multicols}{2}

% Abstract
\noindent\textbf{Abstract} \\[0.5em]
\includegraphics[width=0.9\linewidth]{image11.jpg} \\[0.5em]
\textit{(GATE 2010 IN, Question No.42 – Implementing the above logic diagram Boolean logic using Arduino)}

\vspace{1em}
\noindent\textbf{1. Components}
\begin{table}[H]
\small
\centering
\begin{tabular}{|p{4.2cm}|c|}
\hline
\textbf{Component} & \textbf{Qty} \\
\hline
Arduino UNO Board & 1 \\
USB Cable (Type B) & 1 \\
Push Buttons & 2 \\
LEDs & 1 \\
220$\Omega$ Resistors & 2 \\
Jumper Wires (M-M) & 10 \\
Breadboard & 1 \\
Android Mobile with Arduinodroid App & 1 \\
\hline
\end{tabular}
\caption*{Table 1: List of components used}
\end{table}

\vspace{1em}
\noindent\textbf{2. Setup and Connections}
\begin{enumerate}
    \item Connect push buttons to D2, D3 for A and B.
    \item Add pull-down resistors to each input.
    \item Connect an LED to pin D13 via a 220$\Omega$ resistor.
    \item Use a common ground for buttons and LED.
    \item Power the Arduino via USB and use the Arduinodroid app.
\end{enumerate}

\vspace{1em}
\noindent\textbf{3. Steps for Implementation}
\begin{figure}
    \centering
    \includegraphics[width=0.5\linewidth]{image12.jpg}
    \caption{Enter Caption}
    \label{fig:enter-label}
\end{figure}
\begin{enumerate}
    \item Complete the circuit connections.
    \item Connect Arduino to mobile via USB.
    \item Open Arduinodroid, select board and port.
    \item Open, save, compile and upload the code.
\end{enumerate}

\vspace{1em}
\noindent\textbf{4. Truth Table}
\[
\begin{array}{|c|c|c|}
\hline
X & Y & Z \\
\hline
0 & 0 & 0 \\
0 & 1 & 0 \\
1 & 0 & 1 \\
1 & 1 & 0 \\
\hline
\end{array}
\]

\end{multicols}

\vspace{1em}
\noindent\textbf{5.Implementation}

\noindent
\makebox[\linewidth][l]{%
$\begin{aligned}
Z &= \text{Sum of minterms where output is 1} \\
  &= m_1 + m_2 \\
  &= X\overline{Y} + \overline{X}Y \\
  &= (X \cdot \overline{Y}) + (\overline{X} \cdot Y) \\
  &= \boxed{Z = X \oplus Y}
\end{aligned}$%
}


\vspace{1em}
\noindent\textbf{6. Input and Output Pins}
\begin{flushleft}
\begin{itemize}
    \item \textbf{X (Input)} – D2
    \item \textbf{Y (Input)} – D3
    \item \textbf{Z (Output LED)} – D13
\end{itemize}
\end{flushleft}

\vspace{1em}
\noindent\textbf{7. Arduino Code Link}

\noindent
\url{https://github.com/varalakshmi298/ide/gate/gate_q6.ino}

\vspace{1em}
\begin{figure}[H]
\includegraphics[width=0.5\linewidth]{image12.jpg}
\caption{Implemented Logic Circuit Using Arduino}
\end{figure}

\end{document}
